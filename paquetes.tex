\usepackage[T1]{fontenc}
\usepackage[spanish]{babel}

%para tablas e imagenes
\usepackage{array}
\usepackage{float}
\usepackage{multicol}
%\usepackage{multirow}

%fuentes matematicas y entornos matematicos
\usepackage{amsmath}
\usepackage{amssymb}
\usepackage{amsthm}
\usepackage{xfrac}

%marcar con % si no han comprado la fuente minionpro
%\usepackage[mathlf,textlf,minionint,openg]{MinionPro}
%\usepackage[protrusion=true,expansion=true]{microtype}

%encabezados
\usepackage{fancyhdr}

%colores, mas opciones de tablas y enlaces a ecuaciones
\usepackage{color}
\usepackage{tikz}
\usepackage{makeidx}
\usepackage[linkcolor=blue,colorlinks=true,urlcolor=blue]{hyperref}

\usepackage{titlesec}

\titleformat{\chapter}[display]
{\bfseries\large}
{\titlerule[1pt]%
\vspace{1pt}%
\titlerule
\vspace{1pc}%
\Large\MakeUppercase{\chaptertitlename} \thechapter}
{1pc}
{\titlerule
\vspace{1pc}%
\huge}

\titleformat{\section}
{\Large\bfseries}
{\thesection.}{.5em}{}

\titleformat{\subsection}
{\large \bfseries}
{\thesubsection.}{.5em}{}

\usepackage[left=2cm,right=2cm,top=3cm,bottom=3cm]{geometry}
\parskip=0.5em
\parindent=0em

%\usepackage[onlytext,mathlf,textlf,minionint,openg]{MinionPro}
\usepackage[mathlf,textlf,minionint,openg]{MinionPro}
%\usepackage{lucidabr}
%\usepackage{lucbmath}
%\usepackage[stdmathitalics=true,math-style=iso,lucidasmallscale=true]{lucimatx}
%\usepackage[onlymath=true,scale=.9]{lucimatx}
\usepackage[cal=boondoxo,bb=lucida,bbscaled=.9]{mathalfa}
\usepackage[protrusion=true,expansion=true]{microtype}
\usepackage{textcomp}